\documentclass{article}
\usepackage{graphicx}
\usepackage{amsmath}
\usepackage{polski}
\usepackage[utf8]{inputenc}
\usepackage{hyperref}

\hypersetup{
    colorlinks=true,
    linkcolor=blue,
    filecolor=magenta,
    urlcolor=cyan,
}

\title{Symulowanie Procesów Losowych - Balanced Allocation}
\author{Jakub Kogut}
\date{}

\begin{document}

\maketitle

\section{Wstęp}
Sprawodzanie do zadania domowego 3, zadanie 1. -- \textit{The Power of Two Choices/Balanced Allocation}.
\section{Opis Zadania}
Zadanie polegało na modyfikacji \hyperref{https://github.com/niooch/ballsAndBins}{zadania z 2. listy} w taki sposób, aby zaimplementować algorytm Balanced Allocation. \newline
Celem zadania było wyznaczenie maksymalnej ilości kul w jednej z urn w zależności od parametru $d$.
\begin{enumerate}
    \item $d=1$ - w sposób identyczny do zadania z 2. listy. Wyznaczamy $L_n^{(1)}$.
    \item $d=2$ - algorytm Balanced Allocation. Wyznaczamy $L_n^{(2)}$.
\end{enumerate}
\section{Metodologia}
Podana była ustalona wartość $k=50$ powtórzeń eksperymentu dla każdego $n$. Wartość $n$ miała pochodzić ze zbioru $\{k \times 10^3: k \in \{1, ..., 10^3\}\}$. Dla każdego $n$ wyznaczano średnią z $k$ powtórzeń eksperymentu. Następnie wyznaczano odpowiednio wartość $L_n^{(1)}$ oraz $L_n^{(2)}$.
\section{Wnioski}
W analogii do rozłożenia requestów na serwery, algorytm Balanced Allocation działa lepiej niż algorytm Random Allocation. Wartość $L_n^{(2)}$ rośnie wolniej niż $L_n^{(1)}$ z zwiększającym się $n$. \newline
Na podstawie przeprowadzonych symulacji i wygenerowanych wykresów można wyciągnać natępujące wnioski:
\subsection{Brak Algorytmu Balanced Allocation}
Wartość $L_n^{(1)}$ rośnie asymptotycznie do $\frac{log(n)}{log(log(n))}$ zgodnie z \hyperref[fig:Ln1]{wykresem,} koncentracja wyników w okół wartości średniej jest dość niska, co pokazuje \hyperref[fig:LnA]{ten wykres.} Wynika to jednak z faktu, że wartości wyników są całkowite, a średnia jest jednak ułamkiem.
\subsection{Algorytm Balanced Allocation}
Wartość $L_n^{(2)}$ rośnie asymptotycznie do $\frac{log(log(n))}{log(2)}$ zgodnie z \hyperref[fig:Ln2]{wykresem.}, koncentracja wyników w okół wartości średniej jest podobna do $L_n^{(1)}$, co pokazuje \hyperref[fig:LnB]{ten wykres.}

\section{Podsumowanie}
Jeżeli chodzi o zastosowanie algorytmu Balanced Allocation w praktyce, to zdecydowanie warto go stosować, ponieważ zdecydowanie lepiej radziłby sobie z rozłożeniem requestów na serwery niż algorytm Random Allocation. Wartość $L_n^{(2)}$ rośnie wolniej niż $L_n^{(1)}$ z zwiększającym się $n$. \newline


\section{Wykresy}
\begin{figure}[ht]
    \centering
    \includegraphics[width=0.8\textwidth]{wykresy/Ln1-f1-Ln1Overf1.png}
    \caption{Wykres watosci $\frac{L_n^{(1)}}{f_1}$}
    \label{fig:Ln1}
\end{figure}
\begin{figure}[ht]
    \centering
    \includegraphics[width=0.8\textwidth]{wykresy/Ln2-f2-Ln2Overf2.png}
    \caption{Wykres watosci $\frac{L_n^{(2)}}{f_2}$}
    \label{fig:Ln2}
\end{figure}
\begin{figure}[ht]
    \centering
    \includegraphics[width=0.8\textwidth]{wykresy/LnA.png}
    \caption{Wykres wartości $L_n^{(1)}$}
    \label{fig:LnA}
\end{figure}
\begin{figure}[ht]
    \centering
    \includegraphics[width=0.8\textwidth]{wykresy/LnB.png}
    \caption{Wykres wartości $L_n^{(2)}$}
    \label{fig:Ln1}
\end{figure}
\end{document}
